\chapter{Introduction}

Genome editing is one of the major technological and biological developments of the past few years and will certainly play a major role in the society of the future. In the past few years the advancements in genome editing have sped up enormously and we are looking for ever more precise tools to perform this task. One of the newest tools in our collective toolbox is an enzyme named Cas9. It is a CRISPR enzyme that allows targeting a specific spot on the DNA and cut the DNA at that exact location. This could open the way for a lot of practical uses, for example targeted editing of plant genomes to generate more crops, editing plant genomes to be resistent against pesticides and removing heriditary diseases from embryos. The current issue with Cas9 is that the enzyme is not perfect in its targeting of the DNA, resulting in off-targets which are also cleaved.

To solve this issue it is important to understand why some off-targets are cleaved while others are not. From experiments we have a lot of data from which we can draw empirical conclusions, however a physical model would tell you much more than just a rule-of-thumb. Such a model is being thought out by [SOURCE] . They are attempting to create a kinetic model which predicts the cleavage of Cas9. However since Cas9 targets specific DNA sequences it can be used for more than just cleaving. dCas9 is the dead version of Cas9; it also targets sequences on the DNA but it is incapable of cleavage. This dead enzyme can be used for all kinds of research. Usually a fluorescent molecule is attached to the enzyme so it can be tracked and bound (off-)targets can be identified. In this thesis I will attempt to adapt the cleavage model from [SOURCE] to predict binding and bound off-targets.