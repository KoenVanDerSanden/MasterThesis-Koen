\chapter{Introduction}

Genome editing is one of the major technological and biological developments of the past few years and will certainly play a major role in the society of the future. It will likely allow us to create more productive crops, better drugs, produce more medicine, etc. You name it and most likely genetic engineering can play a part in it! Instead of crafting all our tools and technology ourselves we can now take inspiration from nature itself and even alter nature to better suit it to our needs. In the past few years the advancements in genome editing have sped up enormously and we are looking for ever more precise tools to keep up with our genome editing needs. One of the newest tools in our collective toolbox is an enzyme named Cas9. Naturally Cas9 is found in bacteria where it functions as an immune system and quite a special one at that. It defends bacteria against intruding viruses which inject their DNA into the bacterium. When the DNA enters the bacterium it encounters the Cas9 enzyme. Cas9 is loaded with a strand of RNA which matches the DNA of the virus and therefore allows Cas9 to bind to the virus DNA. When Cas9 is bound it can then cleave the DNA, effectively disabling the virus since making cuts in your DNA turns out to be not very good for your survival chances, as one might suspect. What is surprising is that to load Cas9 with the correct RNA strand to match the virus, the bacterium actually saves parts of the virus DNA in its own DNA. This allows it to load the correct RNA into the Cas9 enzymes but it also ensures all offspring of the bacterium is resistant to any virus containing that piece of DNA as well.

So why is Cas9 so interesting to us? Look at it this way: Cas9 is an enzyme that allows targeting of a specific spot on the DNA and cleave the DNA at that exact location. This precision instrument could open the way for a lot of practical uses, for example targeted editing of plant genomes to generate more crops, editing plant genomes to be resistant against pesticides and removing hereditary diseases from embryos. There is however, as always, one catch: Cas9 is not perfect. Cas9 targets a specific piece of DNA, which is a very effective way to identify and dispose of viruses, assuming they do not mutate. Of course viruses build up mutations all the time, so if Cas9 was perfect in its targeting it would be very easily defeated as an immune system. Nature has solved this by making Cas9 able to cleave not only the perfect matching DNA but also similar pieces of DNA. This is amazing for bacteria but not for us. If we want a precision tool this cleaving of similar DNA pieces (off-targets) is a nuisance at best and a catastrophe at worst. Cleaving of these off-targets could lead to disastrous side-effects if we ever plan to use Cas9 to edit, for example, human embryos.

To solve this issue it is important to understand why some off-targets are cleaved while others are not. A lot of experiments have been done to better understand the off-targeting and people have draw empirical conclusions from the data. However, a physical model would providemore insight and therefore, hopefully, better predictions for the off-targets. The fundamentals for such a physical model have been laid by our group \citep{Misha}. We are attempting to create a kinetic model which predicts the probability a specific DNA sequence is cleaved by Cas9. This work builds on the fundamental model that has been laid out before and tries to apply it to a different related system: dead Cas9 or dCas9 for short.

dCas9 is a modified version of Cas9, it is the same enzyme in essence but is modified in such a way that it is no longer able to cleave. This is achieved by two point mutations in the endonuclease domains. Since it is almost exactly the same enzyme it is likely that the same general model will hold for dCas9 as for Cas9, however since dCas9 is unable to cleave any sequence the specifics of the model must be adjusted. The model as first described in \cite{Misha} predicts the \emph{cleavage} probability of any target sequence, but for dCas9 it is more useful to know the \emph{binding} probability of each target sequence. As is already noted when the cleavage probability is modeled it is expected that 'binding' and 'binding \emph{and} cleaving' are two different things when it comes to (d)Cas9. Therefore the main question this thesis seeks to answer is:

\begin{center}
\emph{"Is it possible to build a more accurate algorithm for predicting the \textbf{binding} of dCas9, based on the physical model from \cite{Misha}, than the existing models based on the empirical observations from experiments, without any underlying physical model?"}
\end{center}

One might ask that if our end goal is genome editing why should we be concerned with the binding probability of dCas9 at all? Would the cleavage probability not be the only parameter of interest? Since dCas9 allows targeting of specific DNA sequences it has many applications outside genome editing. dCas9 can be combined with fluorescent molecules to allow tracking of certain genes, it can be used to block genes from transcribing by occupying space on the DNA or exactly the opposite: use it to activate certain genes by combining it with an activator. These techniques are already demonstrated in \cite{gilbert2014genome}. Some researchers have even been able to turn a gene on or off by shining light on the cell \citep{polstein2015light}. I think \cite{theatlantic} wrote a nice piece to convey what dCas9 potentially could do (emphasis his):

\begin{quote}
"Now, instead of a precise and versatile set of scissors [Cas9], which can cut any gene you want, you have a precise and versatile delivery system, which can \emph{control} any gene you want. You don't just have an editor. You have a stimulant, a muzzle, a dimmer switch, a tracker."
\end{quote}

In short, the main goal of this work is to answer the question if we can build a binding prediction algorithm  for dCas9 based on the foundations of the physical model we laid before. While this is the main goal, since it is based on a physical model which requires an understanding of the binding process, implicitly we also hope to get a better understanding of the way (d)Cas9 behaves and binds to the DNA.

