\chapter{Findings}

Now that we are able to model the system the next step is to get the correct values for the parameters in our model. The exact values of $\delta C, \delta I$ and $\delta PAM$ determine the behaviour of the model, therefore it is essential to get these values as accurate as possible. To obtain these values we will fit the model to data from \cite{PNAS}. This dataset was chosen because of several advantages over other datasets. For one it is very large, it contains all single and double mismatch and on top of that also sequences with more than two mismatches. Furthermore the authors have measured several properties of the enzymes: the on-rate, the off-rate and the occupancy. This allows us to fit on multiple datasets or test our fit for a different experiment. In the end we have tried to fit the minimal model to all three types of data, because for the first two datasets we tried we ran into problems. Still we will report on those problems here since they provided us with insight into the workings of the enzyme and the model.

\section{Dissociation Rate}

The first dataset we tried to fit was the dissociation rate (figure [NUMBER] in \cite{PNAS}). %TODO
The reason for this is that we figured it would be similar to the dissociation constant, for which an expression is already reported in [SOURCE]. %TODO
However it turned out this is not the case. The expression for the dissociation constant as reported in [SOURCE] is: %TODO

\begin{equation}
K_D = 
\end{equation}


% Later, first ceck on- and off-rates
We will start by fitting the minimal model to the occupation data from \cite{PNAS} (figure S2).

For this initial fit we want to keep our options very broad so we do not limit any of the parameters. 